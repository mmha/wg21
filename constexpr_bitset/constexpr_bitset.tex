\documentclass[ebook,11pt,article,a4paper]{memoir}
\usepackage{geometry}
\usepackage{listings,xcolor,beramono}
\usepackage{microtype}
\usepackage[T1]{fontenc}
\usepackage[pdftex, final]{graphicx}
\usepackage{tikz}

\makeatletter
\newenvironment{btHighlight}[1][]
{\begingroup\tikzset{bt@Highlight@par/.style={#1}}\begin{lrbox}{\@tempboxa}}
{\end{lrbox}\bt@HL@box[bt@Highlight@par]{\@tempboxa}\endgroup}

\newcommand\btHL[1][]{%
  \begin{btHighlight}[#1]\bgroup\aftergroup\bt@HL@endenv%
}
\def\bt@HL@endenv{%
  \end{btHighlight}%   
  \egroup
}
\newcommand{\bt@HL@box}[2][]{%
  \tikz[#1]{%
    \pgfpathrectangle{\pgfpoint{1pt}{0pt}}{\pgfpoint{\wd #2}{\ht #2}}%
    \pgfusepath{use as bounding box}%
    \node[anchor=base west, inner ysep=0pt, minimum height=\ht\strutbox+1pt,#1]{\raisebox{1pt}{\strut}\strut\usebox{#2}};
  }%
}
\makeatother

\lstdefinestyle{cppdiff}{
    language={C++},
    basicstyle=\ttfamily\tiny,
    breaklines=true,
    moredelim=**[is][{\btHL[fill=green!30]}]{@}{@},
    moredelim=**[is][{\btHL[fill=red!30]}]{`}{`},
}


\pagestyle{myheadings}

\title{A more constexpr bitset}
\date{}
\setsecnumdepth{subsection}

\begin{document}
\maketitle
\hfill \begin{tabular}[t]{ll}
Document Number:& P1251r0  \\
Date:& 2018-10-04  \\
Audience: & LEWG\\
Reply to: & Morris Hafner  \\
          & hafnermorris@gmail.com  \\
\end{tabular}

\chapter{Motivation}

As of N4762 only the default constructor, the constructor accepting an \texttt{unsigned long long} and \texttt{operator[]} of \texttt{bitset} are marked as constexpr. With the exception of the functions accepting or returning a \texttt{basic\_string}, there is no reason the rest of the class cannot be made \texttt{constexpr}.

The lack of \texttt{constexpr} for most member functions was probably due to the non-trivial destructor of \texttt{bitset::reference}. However, in libc++ it is trivially destructible already. In libstdc++ it is an empty destructor. This behavior should be standardized.

\chapter{Proposed Changes}
Mark every member function except for the \texttt{basic\_string} constructor and \texttt{to\_string} as \texttt{constexpr}. Make \texttt{bitset::reference} trivially destructible and \texttt{constexpr}. Add a constructor accepting a \texttt{basic\_string\_view}.

\chapter{Impact on the Standard}
This proposal is a pure library addition and does not require new language features.

\chapter{Proposed Wording}
Change \textit{19.9.1} of N4762 to the following:

\begin{lstlisting}[style=cppdiff]
#include <string>
#include <iosfwd>   // for istream (27.7.1), ostream (27.7.2), see 27.3.1

namespace std {
  template<size_t N> class bitset;

  // 19.9.4, bitset operators
  template<size_t N>
    @constexpr@ bitset<N> operator&(const bitset<N>&, const bitset<N>&) noexcept;
  template<size_t N>
    @constexpr@ bitset<N> operator|(const bitset<N>&, const bitset<N>&) noexcept;
  template<size_t N>
    @constexpr@ bitset<N> operator^(const bitset<N>&, const bitset<N>&) noexcept;
  template<class charT, class traits, size_t N>
    basic_istream<charT, traits>&
      operator>>(basic_istream<charT, traits>& is, bitset<N>& x);
  template<class charT, class traits, size_t N>
    basic_ostream<charT, traits>&
      operator<<(basic_ostream<charT, traits>& os, const bitset<N>& x);
}
\end{lstlisting}

Change \textit{19.9.2} of N4762 to the following:

\begin{lstlisting}[style=cppdiff]
namespace std {
  template<size_t N> class bitset {
  public:
    // bit reference
    class reference {
      friend class bitset;
      @constexpr@ reference() noexcept;

    public:
      @constexpr@ reference(const reference&) = default;
      `~reference();`
      @constexpr@ reference& operator=(bool x) noexcept;            // for b[i] = x;
      @constexpr@ reference& operator=(const reference&) noexcept;  // for b[i] = b[j];
      @constexpr@ bool operator~() const noexcept;                  // flips the bit
      @constexpr@ operator bool() const noexcept;                   // for x = b[i];
      @constexpr@ reference& flip() noexcept;                       // for b[i].flip();
    };

    // 19.9.2.1, constructors
    constexpr bitset() noexcept;
    constexpr bitset(unsigned long long val) noexcept;
    template<class charT, class traits, class Allocator>
      explicit bitset(
        const basic_string<charT, traits, Allocator>& str,
        typename basic_string<charT, traits, Allocator>::size_type pos = 0,
        typename basic_string<charT, traits, Allocator>::size_type n
          = basic_string<charT, traits, Allocator>::npos,
        charT zero = charT('0'),
        charT one = charT('1'));
    @template<class charT, class traits>@
    @  constexpr explicit bitset(@
    @    const basic_string_view<charT, traits>& str,@
    @    typename basic_string_view<charT, traits>::size_type pos = 0,@
    @    typename basic_string_view<charT, traits>::size_type n@
    @      = basic_string_view<charT, traits>::npos,@
    @    charT zero = charT('0'),@
    @    charT one = charT('1'));@
    template<class charT>
      @constexpr@ explicit bitset(
        const charT* str,
        typename basic_string<charT>::size_type n = basic_string<charT>::npos,
        charT zero = charT('0'),
        charT one = charT('1'));

    // 19.9.2.2, bitset operations
    @constexpr@ bitset<N>& operator&=(const bitset<N>& rhs) noexcept;
    @constexpr@ bitset<N>& operator|=(const bitset<N>& rhs) noexcept;
    @constexpr@ bitset<N>& operator^=(const bitset<N>& rhs) noexcept;
    @constexpr@ bitset<N>& operator<<=(size_t pos) noexcept;
    @constexpr@ bitset<N>& operator>>=(size_t pos) noexcept;
    @constexpr@ bitset<N>& set() noexcept;
    @constexpr@ bitset<N>& set(size_t pos, bool val = true);
    @constexpr@ bitset<N>& reset() noexcept;
    @constexpr@ bitset<N>& reset(size_t pos);
    @constexpr@ bitset<N>  operator~() const noexcept;
    @constexpr@ bitset<N>& flip() noexcept;
    @constexpr@ bitset<N>& flip(size_t pos);

    // element access
    constexpr bool operator[](size_t pos) const;        // for b[i];
    @constexpr@ reference operator[](size_t pos);       // for b[i];

    @constexpr@ unsigned long to_ulong() const;
    @constexpr@ unsigned long long to_ullong() const;
    template<class charT = char,
              class traits = char_traits<charT>,
              class Allocator = allocator<charT>>
      basic_string<charT, traits, Allocator>
        to_string(charT zero = charT('0'), charT one = charT('1')) const;

    @constexpr@ size_t count() const noexcept;
    @constexpr@ constexpr size_t size() const noexcept;
    @constexpr@ bool operator==(const bitset<N>& rhs) const noexcept;
    @constexpr@ bool operator!=(const bitset<N>& rhs) const noexcept;
    @constexpr@ bool test(size_t pos) const;
    @constexpr@ bool all() const noexcept;
    @constexpr@ bool any() const noexcept;
    @constexpr@ bool none() const noexcept;
    @constexpr@ bitset<N> operator<<(size_t pos) const noexcept;
    @constexpr@ bitset<N> operator>>(size_t pos) const noexcept;
  };

  // 19.9.3, hash support
  template<class T> struct hash;
  template<size_t N> struct hash<bitset<N>>;
}
\end{lstlisting}

\end{document}